\documentclass[12pt]{article}

\setlength{\parskip}{1em}

\usepackage[T1]{fontenc}
\usepackage[a4paper, margin=1in]{geometry}
\usepackage{amsfonts}
\usepackage{amsmath}
\usepackage{listings}
\usepackage{xcolor}



\usepackage{hyperref}
\hypersetup{
  colorlinks=true,
  linkcolor=blue,
  urlcolor=blue,
  pdftitle={PC40 Hands-on: UPC}
}

% \definecolor{gray}{rgb}{0.5,0.5,0.5}
% \definecolor{purple}{rgb}{0.58,0,0.82}
\definecolor{bgColor}{rgb}{0.95,0.95,0.92}
\lstdefinestyle{C}{
    backgroundcolor=\color{bgColor},
    commentstyle=\color{gray},
    keywordstyle=\color{magenta},
    numberstyle=\tiny\color{gray},
    stringstyle=\color{purple},
    basicstyle=\footnotesize,
    breakatwhitespace=false,
    breaklines=true,
    captionpos=b,
    keepspaces=true,
    numbers=left,
    numbersep=5pt,
    showspaces=false,
    showstringspaces=false,
    showtabs=false,
    tabsize=2,
    language=C
}

\newcommand{\us}[0]{${\mu}s$}

\title{PC40 Hands-on: UPC}
\author{Adrien Burgun}
\date{Automne 2021}
% \graphicspath{{report/}}

\begin{document}
\maketitle

\begin{abstract}
  For this hands-on assignement, I wanted to measure the performance of the different versions of the code and compare them to measure speedups.
  For this reason, my code diverges slightly from the template that was given to us, as I needed to insert a timing method and avoid refactoring the code.

  You will find in this report listings of the different versions of the code, alongside experimental measurements and commentaries.
\end{abstract}

\section{Simplified 1D Laplace solver}

\subsection{C implementation}

The C implementation of the laplace solver has been slightly modified to time the main loop.
It is single-threaded but the implementation allows for compiler SIMD optimizations.
Settings common to every version (vector size, epsilon, max number of iterations) have been placed in a file called \texttt{settings.h}.

This code has been compiled with \texttt{gcc v4.9.0} and run on the \texttt{mesoshared} server, yielding the following results:

\begin{table}[h]
  \centering\begin{tabular}{|c|c|c|}
    \hline
    Options & Time (avg) & Time (stdev) \\
    \hline
    \texttt{-O0} & 450\us/iter & $\pm$ 10\us/iter \\
    \texttt{-O3} & 140\us/iter & $\pm$ 10\us/iter \\
    \hline
  \end{tabular}
  \caption{Timing results for the C implementation of the 1D Laplace solver (Listing~\ref{lst:laplace2})}
  \label{tab:laplace2}
\end{table}

\lstinputlisting[style=C, label={lst:laplace2}, caption={
  C implementation of the 1D Laplace solver
}]{laplace/ex2.c}

\subsection{Porting the C code to UPC}

The base C implementation was designed to be able to quickly port it to UPC.
A new \texttt{if} statement had to be inserted in the \texttt{for} loop within \texttt{iteration()}.
Additionally, several lines had to only be executed by the thread 0, so additional conditionals were added when needed.
Finally, the \texttt{x}, \texttt{xnew} and \texttt{b} arrays were made shared and \texttt{upc\_barrier} statements were added at the end of \texttt{init}, \texttt{iteration} and \texttt{copy\_array}.

This code was compiled with \texttt{upcc v2.22.0 + gcc v4.2.4} and run on the \texttt{mesoshared} server, yielding the following results:

\begin{table}[h]
  \centering\begin{tabular}{|c|c|c|}
    \hline
    Threads & Time (avg) & Time (stdev) \\
    \hline
    2 & 515.2\us/iter & $\pm$ 3.9\us/iter \\
    3 & 536.2\us/iter & $\pm$ 2.4\us/iter \\
    4 & 312.4\us/iter & $\pm$ 2.0\us/iter \\
    8 & 204.0\us/iter & $\pm$ 2.0\us/iter \\
    16 & 166.2\us/iter & $\pm$ 2.5\us/iter \\
    32 & 150.4\us/iter & $\pm$ 3.2\us/iter \\
    \hline
  \end{tabular}
  \caption{Timing results for the first UPC implementation of the 1D Laplace solver (Listing~\ref{lst:laplace3})}
  \label{tab:laplace3}
\end{table}

When compiled and run with 3 threads, the code runs noticeably slower. For curiosity, I ran the code with 24 and 31 threads and obtained a similar slowdown:

\begin{center}
  Threads = 24, \quad Time = 252.6\us/iter $\pm$ 3.7\us/iter \enspace (expected $\approx 160$ \us/iter) \\
  Threads = 31, \quad Time = 317.6\us/iter $\pm$ 4.4\us/iter \enspace (expected $\approx 150$ \us/iter)
\end{center}

\lstinputlisting[style=C, label={lst:laplace3}, caption={
  First UPC implementation of the 1D Laplace solver
}]{laplace/ex3.upc}

% \lstinputlisting[style=C]{2d_heat/heat_c.c}
\end{document}
